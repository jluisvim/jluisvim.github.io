%///////////////////////////////////////////////////////////////////////////
%	José Luis Vilchis Medina
%	CV in latex: 13/04/2018
%///////////////////////////////////////////////////////////////////////////
\documentclass{article}
\usepackage{natbib}
\usepackage[utf8]{inputenc}
\usepackage[T1]{fontenc}
\usepackage[french]{babel}
\usepackage{geometry}%para los margenes
\geometry{
	a4paper,
	total={170mm,257mm},
	left=20mm,
	top=20mm,
}
\usepackage{multicol}%para las columnas
%\setlength{\columnsep}{1cm}%separacion de las columnas
\renewcommand{\bibsection}{\section*{\centering \underline{Publications}}}
%///////////////////////////////////////////////////////////////////////////
\begin{document}
	\noindent 
	\begin{center}
		{\LARGE \textbf{José Luis Vilchis Medina}}\\[0.3cm]
		\fbox{\parbox{\textwidth}{
				\begin{minipage}[ht]{0.45\textwidth}
					\textbf{\textit{Doctorant en Informatique}}\\
					Parc Scientifique et Technologique de Luminy\\
					163, Avenue de Luminy, Case 901\\
					F-13288 Marseille Cedex 9, FRANCE
				\end{minipage}
				\hspace{4em}
				\begin{minipage}[ht]{0.45\textwidth}
					\begin{flushright}
						\textbf{Email:} \texttt{joselui.vilchismedina@lis-lab.fr}\\
						\textbf{Portable:} \texttt{+33 07 54 25 44 66}\\
						\textbf{\textit{Mise \`{a} jour:}} Juillet 2019
					\end{flushright}
				\end{minipage}
			}
		}
	\end{center}

%***************************************************************************
	\section*{\centering \underline{Position Actuelle}}
	\noindent
	\fbox{\parbox{\textwidth}{
			\begin{minipage}[ht]{0.2\textwidth}
				\footnotesize{\textbf{Sep/2018--Août/19}}
			\end{minipage}
			\hspace{1em}
			\begin{minipage}[ht]{0.52\textwidth}
				\begin{center}
					\textbf{Aix-MArseille Université}\\
					\textbf{L}aboratoire d'\textbf{I}nformatique et \textbf{S}ystèmes (LIS)\\
					\textit{Licence Informatique}\\
					\textbf{Attaché Temporaire d'Enseignement et de Recherche (ATER)}
					
				\end{center}
			\end{minipage}
			\hspace{1em}
			\begin{minipage}[ht]{0.2\textwidth}
				{\flushright \textbf{Marseille, France}}
			\end{minipage}
		}
	}

%***************************************************************************
\section*{\centering \underline{Thématiques de Recherche}}
	\noindent 
	\fbox{\parbox{\textwidth}{
			\begin{multicols}{2}
					\begin{description}
						\item[$\bullet$] Représentation des Connaissances
						\item[$\bullet$] Raisonnement Non-monotone
						\item[$\bullet$] Systèmes Autonomes
					\end{description}
					\columnbreak
					\begin{description}
						\item[$\bullet$] Raisonnement Incertain
						\item[$\bullet$] Logique des Défauts
						\item[$\bullet$] Systèmes Embarqués
					\end{description}
			\end{multicols}
		}
	}
%***************************************************************************
 	\section*{\centering \underline{Formation}}
	\noindent
	\fbox{\parbox{\textwidth}{
 			\begin{minipage}[ht]{0.2\textwidth}
 				\footnotesize{\textbf{Septembre/2015--Décembre/2018}}
 			\end{minipage}
 			\hspace{1em}
 			\begin{minipage}[ht]{0.52\textwidth}
 				\begin{center}
 					\underline{\textbf{Doctorant en Informatique}}\\
 					\textbf{L}aboratoire d'\textbf{I}nformatique et \textbf{S}ystèmes (LIS)\\
					Equipe CAlcul NAturel\\[0.15cm]
			\textbf{Sujet de recherche:}\\
					\textit{Modeling of Resilient System in Default Logic. Application to Solar Power UAV.}\\[0.15cm]
					\textbf{Directeur:}\\
					Pierre SIEGEL, Andrei DONCESCU\\
 				\end{center}
 			\end{minipage}
 			\hspace{1em}
 			\begin{minipage}{0.2\textwidth}
				\begin{flushright}
					\textbf{Marseille, France}
				\end{flushright}
 			\end{minipage}
 		}
	}
%***************************************************************************
	\fbox{\parbox{\textwidth}{
 			\begin{minipage}[ht]{0.2\textwidth}
 				\footnotesize{\textbf{Septembre 2015}}
 			\end{minipage}
 			\hspace{1em}
 			\begin{minipage}[ht]{0.52\textwidth}
 				\begin{center}
 					\underline{\textbf{M.sC. Ingénieur Génie \'{E}lectrique et Automatique}}\\
 					INP-ENSEEIHT\\[0.15cm]
 					\textbf{Option:}\\
 					Commande, Décision et Informatique des Systèmes Critiques\\[0.15cm]
 					\textbf{Projet de Fin d’études:}\\
 					\textit{ Conception d’un démonstrateur électronique versatile pour la mesure de déplacements par réinjection optique dans une diode laser, avec contrôle du faisceau émis}\\[0.15cm]
 					\textbf{Tutor:}\\
 					Julien PERCHOUX, Antonio LUNA ARRIAGA\\
 				\end{center}
 			\end{minipage}
 			\hspace{1em}
 			\begin{minipage}{0.2\textwidth}
				\begin{flushright}
					\textbf{Toulouse, France}
				\end{flushright}
 			\end{minipage}
 		}
	}
%---------------------------------------------------------------------------
	\fbox{\parbox{\textwidth}{
			\begin{minipage}[ht]{0.2\textwidth}
				\footnotesize{\textbf{Août 2012}}
 			\end{minipage}
 			\hspace{1em}
 			\begin{minipage}[ht]{0.52\textwidth}
 				\begin{center}
 					\underline{\textbf{B.S., Ingénieur en Électronique}}\\ 
 					Universidad Autónoma de Baja California\\[0.15cm]
 					\textbf{Option:}\\
					Contrôle de Systèmes - \textit{Mention Honorifique}\\[0.15cm]
 					\textbf{Projet de Fin d’études:}\\
 					Développement des systèmes embarques et ses applications.\\[0.15cm]
 				\end{center}
 			\end{minipage}
 			\hspace{1em}
 			\begin{minipage}{0.2\textwidth}
 				\begin{flushright}
					\textbf{Ensenada, Mexique}
 				\end{flushright}
 			\end{minipage}
 		}
 	}
%***************************************************************************
\iffalse
	\section*{\centering \underline{Enseignement}}
	\noindent	
	\fbox{\parbox{\textwidth}{
			\begin{minipage}[ht]{0.2\textwidth}
				\footnotesize{\textbf{Octobre -- Décembre 2017}}
			\end{minipage}
			\hspace{1em}
			\begin{minipage}[ht]{0.52\textwidth}
				\begin{center}
					\textbf{Systèmes Embarqués, TD/TP (20h)--M1}\\
					Aix-Marseille Université\\[0.15cm]
				\end{center}
			\end{minipage}
			\hspace{1em}
			\begin{minipage}{0.2\textwidth}
				\begin{flushright}
					\textbf{Marseille, France}
				\end{flushright}
			\end{minipage}
		}
	}
%---------------------------------------------------------------------------
	\fbox{\parbox{\textwidth}{
			\begin{minipage}[ht]{0.2\textwidth}
				\footnotesize{\textbf{Septembre -- Novembre 2016}}
			\end{minipage}
			\hspace{1em}
			\begin{minipage}[ht]{0.52\textwidth}
				\begin{center}
					\textbf{Systèmes Embarqués, TD/TP (20h)--M1}\\
					Aix-Marseille Université\\[0.15cm]
				\end{center}
			\end{minipage}
			\hspace{1em}
			\begin{minipage}{0.2\textwidth}
				\begin{flushright}
					\textbf{Marseille, France}
				\end{flushright}
			\end{minipage}
		}
	}
%---------------------------------------------------------------------------
		\fbox{\parbox{\textwidth}{
			\begin{minipage}[ht]{0.2\textwidth}
				\footnotesize{\textbf{Février--Avril 2016}}
			\end{minipage}
			\hspace{1em}
			\begin{minipage}[ht]{0.52\textwidth}
				\begin{center}
					\textbf{Programmation Synchrone de Micro-contrôleurs, TP(30h)--L1}\\
					Aix-Marseille Université\\[0.15cm]
				\end{center}
			\end{minipage}
			\hspace{1em}
			\begin{minipage}{0.2\textwidth}
				\begin{flushright}
					\textbf{Marseille, France}
				\end{flushright}
			\end{minipage}
		}
	}

%***************************************************************************
	\section*{\centering \underline{Expérience Professionnelle}}
	\noindent
		\fbox{\parbox{\linewidth}{
				\begin{minipage}[ht]{0.2\textwidth}
					\footnotesize{\textbf{Octobre 2012 -- Juillet 2013}}
				\end{minipage}
				\hspace{1em}
				\begin{minipage}[ht]{0.52\textwidth}
					\begin{center}
						\textbf{Ingénieur Logiciel }\\
						Navico\\[0.15cm]
						\textbf{Projets:}\\
						Conception et optimisation des algorithmes d’acquisition de données pour des signaux GPS et GLONASS dans un bras robotisé pour la validation
						des antennes sur LabVIEW/TestStand.\\[0.15cm]
						Conception et validation d’une base de données (LAN) pour l’inventaire des équipements de test sur LabVIEW.
					\end{center}
				\end{minipage}
				\hspace{1em}
				\begin{minipage}{0.2\textwidth}
					\begin{flushright}
						\textbf{Ensenada, Mexique}
					\end{flushright}
				\end{minipage}
			}
		}
%---------------------------------------------------------------------------
	\fbox{\parbox{\linewidth}{
				\begin{minipage}[ht]{0.2\textwidth}
					\textbf{Juin 2012 -- Septembre 2012}
				\end{minipage}
				\hspace{1em}
				\begin{minipage}[ht]{0.52\textwidth}
					\begin{center}
						\textbf{Technicien en électronique}\\
						Navico\\[0.15cm]
						Coordination et réparation de produits électroniques marins (ordinateur pour les systèmes de navigation, ordinateur pour le pilote automatique,
						instruments numériques d’affichage et plus), avec suivi en ligne pour la validation des garanties de rénovation.
					\end{center}
				\end{minipage}
				\hspace{1em}
				\begin{minipage}{0.2\textwidth}
					\begin{flushright}
						\textbf{Ensenada, Mexique}
					\end{flushright}
				\end{minipage}
			}
		}
		\fi
%***************************************************************************
	\section*{\centering \underline{Autres Projets}}
	\noindent
		\fbox{\parbox{\linewidth}{
				\begin{minipage}[ht]{0.2\textwidth}
					\footnotesize{\textbf{Février 2014 - Mars 2014}}
				\end{minipage}
				\hspace{1em}
				\begin{minipage}[ht]{0.52\textwidth}
					\begin{center}
						\textbf{Stage -- Projet Long}\\
						LAAS-RAP\\[0.15cm]
						\textbf{Projet:}\\
						Localisation binaural des sons de sources multiples en Robotique. Le projet a été developpé en langage C. (Audition en robotique)\\
					\end{center}
				\end{minipage}
				\hspace{1em}
				\begin{minipage}{0.2\textwidth}
					\begin{flushright}
						\textbf{Toulouse, France}
					\end{flushright}
				\end{minipage}
			}
		}
%---------------------------------------------------------------------------
		\fbox{\parbox{\linewidth}{
				\begin{minipage}[ht]{0.2\textwidth}
					\footnotesize{\textbf{Janvier 2014 - Février 2014}}
				\end{minipage}
				\hspace{1em}
				\begin{minipage}[ht]{0.52\textwidth}
					\begin{center}
						\textbf{Stage -- Projet Long Industriel}\\
						INP-ENSEEIHT\\[0.15cm]
						\textbf{Projet:}\\
						Projet en collaboration avec Continental AUTOMOTIVE FRANCE, pour l’étude des stratégies d’hybridation pour un véhicule routier. Modélisation
						et contrôle des deux chaînes.\\
					\end{center}
				\end{minipage}
				\hspace{1em}
				\begin{minipage}{0.2\textwidth}
					\begin{flushright}
						\textbf{Toulouse, France}
					\end{flushright}
				\end{minipage}
			}
		}
		
		\iffalse
%***************************************************************************
	\section*{\centering \underline{Compétences}}
	\noindent \fbox{\parbox{\textwidth}{
			%\begin{center}
				\centering{\textbf{Mathématiques:}}\\
					Mathématiques appliquées, Analyse réelle et complexe, Théorie de la mesure, Géométrie différentielle, Algèbre et Probabilités\\
					\centering{\textbf{Théorie du contrôle et de l’ingénierie:}}\\ 
					Théorie des systèmes linéaire et non linéaire, Feedback, Systèmes de structure variables, Control distributed et Intelligent, Optimisation continue, Systèmes physiques, Commande optimisées des systèmes\\
					\centering{\textbf{Communications et Traitement du Signal:}}\\
					Probabilités, Variables aléatoires, Processus Stochastiques, Théorie de l’information, Estimation, Réseaux\\
					\textbf{Informatique et Ingénierie:} Model Checking (automatisé, distribué, hybride), vérification de code, Logiciel à base de composants réutilisables
				}

			}
		\noindent
			\fbox{\parbox{\textwidth}{
				\centering{\textbf{Matérielles et Logiciels:}}\\
				\centering{\textbf{Électronique Analogique et Numérique:}}\\
				Implémentations d’amplificateurs continus et commutés avec transistor Bipolaires et FET, Op-Amp,Modulateurs, Convertisseurs et Filtres.\\
				\centering{\textbf{Outils de conception assistées par ordinateur:}}\\
				OrCAD, NI Multisim, SPICE, ISIS-Proteus.\\
				\centering{\textbf{Systèmes Embarqués et Systèmes en Temps Réel:}}\\
				Conception et développement logiciels et matériel avec plusieurs plates-formes MCU et DSP(par exemple, Texas Instruments DSP, Atmel ATmega MCU, Microchip PIC MCU, et autres)\\
				\centering{\textbf{Instrumentation, Contrôle, Acquisition de données, Test et Mesure:}}\\
				Simulink, MATLAB Coder, Simulink Coder, LabVIEW et autres matériels d’acquisition de données multifonction de National Instruments (Logiciels et Matériel) (e.g., PCI, USB-	6008, USB-6251 et plus), équipements bench-top Hewlett-Packard et Agilent\\
				\centering{\textbf{Programmation:}}\\
				C, C++,Python, G (LabVIEW), GNUmake,AppleScript, RPL\ldots\\
				\centering{\textbf{Analyse numérique:}}\\
				MATLAB,Maple,Mathematica,Octave,OpenCV\\
				\centering{\textbf{Bureau d’Édition et Logiciels de productivité:}}\\
				Vim, Emacs, Eclipse, \LaTeX, Microsoft Office, OpenOffice, LibreOffice\\
				\centering{\textbf{Systèmes d’exploitation:}}\\
				Microsoft Windows, Apple OS X, Linux et d’autres variantes d’UNIX
			}
		}
		\noindent \fbox{\parbox{\textwidth}{
					\centering{\textbf{Languages:}}\\[0.2cm]
					\textbf{Espagnol:} Langue Maternelle, \textbf{Français:} TCF 5/6, \textbf{Anglais:} TOEFL 583/677, \textbf{Portugaise:} A1
				}
		}
%***************************************************************************
	\section*{\centering \underline{Prix/Distinctions}}
		\noindent \fbox{\parbox{\textwidth}{
			\centering{\textbf{Haute réussite scolaire par le Centre national d’évaluation (CENEVAL)}}\\[0.2cm]
			Universidad Autónoma de Baja California, Promotion 2012\\
			Génie Électronique, Mention: Très Bien
		}
	}
	\noindent \fbox{\parbox{\textwidth}{
			\centering{\textbf{Bourse d’echange Mexico-France Ingénieurs Technologie (MEXFITEC)}}\\[0.2cm]
			Génie Électronique et Traitement du Signal en 2ème année\\
			À l’INP-ENSEEIHT
		}	
	}
	\fi
%***************************************************************************
	\section*{\centering \underline{Présentations}}
	\noindent 
	\fbox{\parbox{\textwidth}{
			\begin{minipage}[ht]{0.2\textwidth}
				\footnotesize{\textbf{Janvier 2017}}
			\end{minipage}
			\hspace{1em}
			\begin{minipage}[ht]{0.52\textwidth}
				\begin{center}
					\textbf{L'École Jeunes Chercheurs et Chercheuses en Informatique Mathématique}\\
					ENS de Lyon\\
				\end{center}
			\end{minipage}
			\hspace{1em}
			\begin{minipage}{0.2\textwidth}
				\begin{flushright}
					\textbf{Lyon, France}
				\end{flushright}
			\end{minipage}
		}
	}
	\fbox{\parbox{\textwidth}{
			\begin{minipage}[ht]{0.2\textwidth}
				\footnotesize{\textbf{Juin 2017}}
			\end{minipage}
			\hspace{1em}
			\begin{minipage}[ht]{0.52\textwidth}
				\begin{center}
					\textbf{Semainière CANA, LIS}\\[0.15cm]
					\textbf{Titre:}\\
					Modeling a Resilient System using Non-monotonic Logic.\\[0.15cm]
					\textit{Campus Luminy}\\[0.15cm]
				\end{center}
			\end{minipage}
			\hspace{1em}
			\begin{minipage}{0.2\textwidth}
				\begin{flushright}
					\textbf{Marseille, France}
				\end{flushright}
			\end{minipage}
		}
	}
	\fbox{\parbox{\textwidth}{
			\begin{minipage}[ht]{0.2\textwidth}
				\footnotesize{\textbf{Juillet 2017}}
			\end{minipage}
			\hspace{1em}
			\begin{minipage}[ht]{0.52\textwidth}
				\begin{center}
					\textbf{Journées Francophones sur la Planification, la Décision et l'Apprentissage pour la Conduite de Systèmes}\\
					PFIA 2017\\[0.15cm]
				\end{center}
			\end{minipage}
			\hspace{1em}
			\begin{minipage}{0.2\textwidth}
				\begin{flushright}
					\textbf{Caen, France}
				\end{flushright}
			\end{minipage}
		}
	}
	\fbox{\parbox{\textwidth}{
		\begin{minipage}[ht]{0.2\textwidth}
			\footnotesize{\textbf{Novembre 2017}}
		\end{minipage}
		\hspace{1em}
		\begin{minipage}[ht]{0.52\textwidth}
			\begin{center}
				\textbf{Semainière LIRICA, LIS}\\[0.15cm]
				\textbf{Titre:}\\
				Non-monotonic Reasoning and Uncertain Decision-Making: Application to an Autonomous Glider.\\[0.15cm]
				\textit{FRUMAM, Campus St. Charles}\\[0.15cm]
			\end{center}
		\end{minipage}
		\hspace{1em}
		\begin{minipage}{0.2\textwidth}
			\begin{flushright}
				\textbf{Marseille, France}
			\end{flushright}
		\end{minipage}
	}
}
%***************************************************************************
	\section*{\centering \underline{Langues}}
		\noindent \fbox{\parbox{\textwidth}{
		
		\begin{center}
			\begin{minipage}[ht]{0.2\linewidth}
				\textbf{Espagnol}\\
				Langue Maternelle
			\end{minipage}
			\hspace{4em}
			\begin{minipage}[ht]{0.2\linewidth}
				\textbf{Anglais}\\
				Niveau Avancé
			\end{minipage}
			\begin{minipage}[ht]{0.2\linewidth}
				\textbf{Français}\\
				Niveau Avancé
			\end{minipage}
			\begin{minipage}[ht]{0.2\linewidth}
				\textbf{Portugais}\\
				Niveau Basique
			\end{minipage}
		\end{center}
		
		}
	}
%***************************************************************************
\section*{\centering \underline{Références}}
	\noindent 
	\fbox{\parbox{\textwidth}{
			\centering
			\begin{multicols}{2}
				\textbf{Pierre SIEGEL}\\
				pierre.siegel@lis-lab.fr\\
				\columnbreak
				\textbf{Laboratoire d'Informatique et Systèmes}\\
				Marseille, France
		\end{multicols}
		}
	}
	\fbox{\parbox{\textwidth}{
			\centering
			\begin{multicols}{2}
				\textbf{Andrei DONCESCU}\\
				andrei.doncescu@laas.fr\\
				\columnbreak
				\textbf{LAAS--CNRS}\\
				Toulouse, France
		\end{multicols}
		}
	}

%***************************************************************************
\newpage
%	\printbibliography[title={Publications}]
	\bibliographystyle{unsrt}
	\bibliography{../mybiblio}

	\nocite{*}

%***************************************************************************
\end{document}
